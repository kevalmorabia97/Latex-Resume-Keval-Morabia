\documentclass[]{Resume}
\usepackage{fancyhdr}
\usepackage{hyperref}

\pagestyle{fancy}
\fancyhf{}
\renewcommand{\headrulewidth}{0pt}

\hypersetup{%
 	colorlinks=false,% hyperlinks will be black
 	pdfborderstyle={/S/U/W 0.25}% border style will be underline of width 0.25pt
}
 
\begin{document}

%%%%%%%%%%%%%%%%%%%%%%%%%%%%%%%%%%%%%%
%     TITLE NAME
%%%%%%%%%%%%%%%%%%%%%%%%%%%%%%%%%%%%%%
\namesection{Keval Morabia}{
    % xxx-xxx-xxxx -> Phone Number
	\normalfont xxx-xxx-xxxx \pt Urbana, Illinois 61801 \pt \href{mailto:kevalmorabia97@gmail.com}{kevalmorabia97@gmail.com} \\
	
	\normalfont \href{https://linkedin.com/in/kevalmorabia97}{\textbf{linkedin}.com/in/kevalmorabia97} \pt \href{https://github.com/kevalmorabia97}{\textbf{github}.com/kevalmorabia97}
}

\descript{}

%%%%%%%%%%%%%%%%%%%%%%%%%%%%%%%%%%%%%%
%     EDUCATION
%%%%%%%%%%%%%%%%%%%%%%%%%%%%%%%%%%%%%%
\section{Education}
\hrulefill

\subsection{University of Illinois at Urbana-Champaign \hfill \normalfont Urbana, IL}
\position{Masters degree in Computer Science (GPA: 4/4) \normalfont with specialization in Artificial Intelligence}{Aug '19 - Dec '20}
\pt \textbf{Coursework:} Computer Vision, Advanced Information Retrieval, Machine Learning in NLP, Autonomous Vehicles, UI Design \\
\pt \textbf{Teaching Experience:} Natural Language Processing, Discrete Structures
\sectionsep

\subsection{Birla Institute of Technology and Science (BITS), Pilani \hfill \normalfont Hyderabad, India}
\position{Bachelor of Engineering (Hons.) in Computer Science (GPA: 9.72/10) \normalfont}{Aug '15 - May '19}
% \pt \textbf{Electives:} Artificial Intelligence, Machine Learning, Information Retrieval, Data Mining, Number Theory, Cryptography \\
% \pt \textbf{Teaching Experience:} Data Mining, Object-oriented Programming
\sectionsep

%%%%%%%%%%%%%%%%%%%%%%%%%%%%%%%%%%%%%%
%     EXPERIENCE
%%%%%%%%%%%%%%%%%%%%%%%%%%%%%%%%%%%%%%
\section{Experience} 
\hrulefill 

\subsection{Amazon Web Services (AWS) \hfill \normalfont Greater Seattle Area, WA}
\position{Software Development Engineer Intern}{May '20 - Aug '20}
\pt Deployed \textbf{Java APIs} to \textbf{AWS} cloud for providing preview of tasks to be performed by technicians in Amazon data centers \\
\pt Analyzed complex BPMN Workflows extracted from \textbf{Amazon Dynamo DB} to identify expected order of task execution \\
\pt Tested code with Unit, Integration and Load testing using \textbf{JUnit}, \textbf{Mockito}, and \textbf{TestNG}
\sectionsep

\subsection{University of Illinois at Urbana-Champaign \hfill \normalfont Urbana, IL}
\position{Research Assistant with \href{http://www.forwarddatalab.org/kevinccchang}{\textbf{Prof. Kevin Chang}}}{Aug '19 - May '20}
\pt Experimented a \textbf{Visual Attention-based} Model in \textbf{PyTorch} for novel Webpage Object Detection formulation \\
\pt Utilized contextual information using visual features of ordered web elements extracted using \textbf{Resnet18} \\  
\pt Created largest public labeled dataset of 7.7k product webpage screenshots \\
\pt Achieved \textbf{95\% accuracy} for product Price detection (8.5\% above Fast R-CNN) and interpreted Attention Visualizations
\sectionsep

\subsection{Microsoft Research \hfill \normalfont Bengaluru, India}
\position{Research Intern}{Jan '19 - Jul '19}
\pt Implemented \textbf{Graph RNNs} in \textbf{TensorFlow} to Learn Embeddings for \textbf{300k} entities in a heterogeneous graph \\
\pt Collaborated with a \textbf{team of 10} to design a novel Deep Neural Net architecture for \textbf{recommending messages} in MSTeams \\
\pt Outperformed Graph Convolutional Networks on \textbf{5} benchmark datasets for rating prediction and ranking tasks
\sectionsep

\subsection{BITS Pilani \hfill \normalfont Hyderabad, India}
\position{Research Assistant}{Jan '18 - Dec '18} 
\pt Analyzed \textbf{Twitter} Stream for \textbf{Event Detection} and achieved a \textbf{precision of 88.1\%} (absolute improvement of 8\%) \\
\pt Segmented tweets and hash-tags; applied Jarvis-Patrick clustering; summarized newsworthy events 
\sectionsep

%%%%%%%%%%%%%%%%%%%%%%%%%%%%%%%%%%%%%%
%     PUBLICATIONS
%%%%%%%%%%%%%%%%%%%%%%%%%%%%%%%%%%%%%%
\section{Publications} 
\hrulefill 

\textbf{Morabia, K.}, Bhanu Murthy, N. L., Malapati, A., \& Samant, S. (\textbf{2019}, June). \textbf{SEDTWik: Segmentation-based Event Detection from Tweets using Wikipedia}. In Proceedings of the 2019 Conference of the North American Chapter of the Association for Computational Linguistics \textbf{(NAACL)}: Student Research Workshop (pp. 77-85) \href{https://www.aclweb.org/anthology/N19-3011/}{[\textbf{Paper}]} \href{https://github.com/kevalmorabia97/SEDTWik-Event-Detection-from-Tweets}{[\textbf{Code}]}
\sectionsep

%%%%%%%%%%%%%%%%%%%%%%%%%%%%%%%%%%%%%%
%     PROJECTS
%%%%%%%%%%%%%%%%%%%%%%%%%%%%%%%%%%%%%%
\section{Project Highlights}
\hrulefill

%%%%% SIMPLIFIED FORMAT %%%%%
% \pt Attention-based Joint Detection of Object and Semantic Part in \textbf{PyTorch} \href{https://arxiv.org/abs/2007.02419}{[\textbf{arXiv:2007.02419}]} \href{https://github.com/kevalmorabia97/Object-and-Semantic-Part-Detection-pyTorch}{[\textbf{Code}]}
% \pt Built an application in \textbf{JavaFX} for \textbf{Frequent Patterns \& Association Rule Mining} using Apriori algorithm \href{https://github.com/kevalmorabia97/FPARM-Frequent-Patterns-and-Association-Rule-Miner}{[\textbf{Code}]}

%%%%% ELABORATED FORMAT %%%%%
\position{Building an Autonomous Vehicle}{Sep '20 - Nov '20}
\pt Collaborated in a \textbf{team of 6} to control a Polaris GEM e2 Autonomous Vehicle containing a LiDAR, Radar, and camera \\
\pt Programmed WASD keyboard keys to accelerate and steer vehicle using a PID controller and ROS Python commands \\
\pt Deployed pre-trained \textbf{YOLOv3} to stop vehicle on detecting a pedestrian
\sectionsep

\position{Attention-based Joint Detection of Object and Semantic Part \href{https://arxiv.org/abs/2007.02419}{[\textbf{arXiv:2007.02419}]}  \href{https://github.com/kevalmorabia97/Object-and-Semantic-Part-Detection-pyTorch}{[\textbf{Code}]}}{Feb '20 - May '20}
\pt Implemented a \textbf{PyTorch} model for simultaneous Object and Part detection by combining 2 \textbf{Faster-RCNN} models \\
\pt Applied a novel \textbf{Attention-based feature fusion} to learn enhanced representation for Object and Part proposals \\
\pt Improved mean Average Precision for Objects and Parts on PASCAL-Part dataset
\sectionsep

%%%%%%%%%%%%%%%%%%%%%%%%%%%%%%%%%%%%%%
%     SKILLS
%%%%%%%%%%%%%%%%%%%%%%%%%%%%%%%%%%%%%%
\section{Skills} 
\hrulefill

\pt \textbf{Programming:} Python, Java, C, C++, LaTex, SQL, HTML, CSS, JavaScript \\
\pt \textbf{Other Technologies:} PyTorch, TensorFlow, AWS, Scikit-learn, GIT, Flutter, Android Studio, JUnit, Mockito

\sectionsep

%%%%%%%%%%%%%%%%%%%%%%%%%%%%%%%%%%%%%%
%     LEADERSHIP AND HONORS
%%%%%%%%%%%%%%%%%%%%%%%%%%%%%%%%%%%%%%
\section{Leadership and Honors} 
\hrulefill 

\begin{minipage}[t]{.8\textwidth}
	\pt Awarded \textbf{Merit Scholarship} (\textbf{Top 1\%} out of 700 students at BITS Pilani, Hyderabad)
\end{minipage}%
\begin{minipage}[t]{.2\textwidth}
	\hfill Aug '15 - May '19
\end{minipage}

\begin{minipage}[t]{.8\textwidth}
	\pt Organized cultural dance events as \textbf{President} of 150-membered Gujarati Association - BITS Pilani, Hyderabad; attended by over 1000 students; raised INR 30k fund
\end{minipage}%
\begin{minipage}[t]{.2\textwidth}
	\hfill Jul '17 - May '18
\end{minipage}

%\sectionsep

%%%%%%%%%%%%%%%%%%%%%%%%%%%%%%%%%%%%%%	
\end{document}  \documentclass[]{article}
